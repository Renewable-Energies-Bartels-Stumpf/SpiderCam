\section{Raumbegrenzung}
		Bevor eine Längenänderung ermittelt werden kann, muss zunächst garantiert werden, dass alle Punkte der Bewegung erreicht werden können. Da zuvor zwei der drei Eckpunkte auf die x- Achse gelegt wurden, kann nun einfach ermittelt werden, ob alle Punkte unterhalb der Verbindungsgeraden y \textsubscript{1} und y \textsubscript{2} liegen.
		Dafür werden folgende Geradengleichungen aufgestellt:
		\begin{align}
			y \textsubscript{1} = m \textsubscript{1} * x + b \textsubscript{1}\\
			y \textsubscript{2} = m \textsubscript{2} * x + b \textsubscript{2}
		\end{align}
		Mit b\textsubscript{1}=0 und m\textsubscript{1}=$\frac{\Delta y}{\Delta x}$ ergibt sich y\textsubscript{1} zu:
		\begin{align}
			y\textsubscript{1} = \frac{c \textsubscript{y}-a \textsubscript{y}}{c \textsubscript{x}-a \textsubscript{x}} * x
		\end{align}
		Mit m\textsubscript{2}=$\frac{\Delta y}{\Delta x}$ ergibt sich y\textsubscript{2} zu:
		\begin{align}
			y\textsubscript{2} = \frac{c\textsubscript{y}-b\textsubscript{y}}{c\textsubscript{x}-b\textsubscript{x}}*x+b\textsubscript{2}
		\end{align}
		Da die Koordinaten von Eckpunkt B bekannt sind, kann b\textsubscript{2} mit Hilfe von m\textsubscript{2} ermittelt werden.
		\begin{align}
			m\textsubscript{2} = \frac{y\textsubscript{2}-y\textsubscript{1}}{x\textsubscript{2}-x\textsubscript{1}}
		\end{align}
		Mit b\textsubscript{2}=y\textsubscript{2},x\textsubscript{2}=0 und m\textsubscript{2}=$\frac{c\textsubscript{y}-b\textsubscript{y}}{c\textsubscript{x}-b\textsubscript{x}}$ folgt:
		\begin{align}
			b\textsubscript{2} = \frac{(c\textsubscript{y}-c\textsubscript{x})*(c\textsubscript{y}-b\textsubscript{y})}{c\textsubscript{x}-b\textsubscript{x}}
		\end{align}
		Da nun beide Geradengleichungen definiert sind, kann überprüft werden, ob alle Punkte der Bewegung folgende Bedingungen erfüllen:
		\begin{align}
			p\textsubscript{x}<c\textsubscript{x} 
			\land 	p\textsubscript{y}<y\textsubscript{1}(p\textsubscript{x}) 
			\lor
			p\textsubscript{x} \geq c\textsubscript{x} 
			\land 	p\textsubscript{y}<y\textsubscript{2}(p\textsubscript{x})
		\end{align}
		Für die räumliche Begrenzung einer Bewegung im dreidimensionalen Raum kann die zuvor aufgestellte Ungleichung mit folgender Bedingung erweitert werden:
		\begin{align}
			P\textsubscript{z}<A\textsubscript{z}\land P\textsubscript{z}>0
		\end{align}
		Damit muss nun jeder Punkt einer Bewegung folgende Bedingung erfüllen:
		\begin{align}
			P\textsubscript{x}<C\textsubscript{x} 
			\land 	P\textsubscript{y}<y\textsubscript{1}(P\textsubscript{x})
			\land	P\textsubscript{z}<A\textsubscript{z}
			\land 	P\textsubscript{z}>0 
			\lor
			P\textsubscript{x} \geq C\textsubscript{x} 
			\land 	P\textsubscript{y}<y\textsubscript{2}(P\textsubscript{x})
			\land	P\textsubscript{z}<A\textsubscript{z}
			\land 	P\textsubscript{z}>0
		\end{align}
		Um die Begrenzung einer Plattform im dreidimensionalen Raum aufzustellen muss die Ungleichung erneut erweitert werden. Es müssen nun anstelle eines Punktes, welcher die Bedingung erfüllt, alle Punkte, welche die Plattform bilden, die Bedingung erfüllen. Es wird vereinfacht überprüft, ob alle Eckpunkte der Plattform die Bedingung erfüllen. Daraus folgt, dass alle Punkte der Plattform ebenfalls die Bedingung erfüllen, da alle Extrempunkte der Plattform überprüft wurden.Für n=1,2,3 muss gelten:
		\begin{align}
		P\textsubscript{n,x}<C\textsubscript{x} 
		\land 	P\textsubscript{n,y}<y\textsubscript{1}(P\textsubscript{n,x})
		\land	P\textsubscript{n,z}<A\textsubscript{z}
		\land 	P\textsubscript{n,z}>0
		\lor
		P\textsubscript{n,x} \geq C\textsubscript{x} 
		\land 	P\textsubscript{n,y}<y\textsubscript{2}(P\textsubscript{n,x})
		\land	P\textsubscript{n,z}<A\textsubscript{z}
		\land 	P\textsubscript{n,z}>0
		\end{align}
	\pagebreak