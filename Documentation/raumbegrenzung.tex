\section{Raumbegrenzung}
		Bevor eine Längenänderung ermittelt werden kann, muss zunächst garantiert werden, dass alle Punkte der Bewegung erreicht werden können. Da zuvor zwei der drei Eckpunkte auf die x- Achse gelegt wurden, kann nun einfach ermittelt werden, ob alle Punkte unterhalb der Verbindungsgeraden y \textsubscript{1} und y \textsubscript{2} liegen.
		Dafür werden folgende Geradengleichungen aufgestellt:
		\begin{align}
			y \textsubscript{1} = m \textsubscript{1} * x + b \textsubscript{1}\\
			y \textsubscript{2} = m \textsubscript{2} * x + b \textsubscript{2}
		\end{align}
		Mit b\textsubscript{1}=0 und m\textsubscript{1}=$\frac{\Delta y}{\Delta x}$ ergibt sich y\textsubscript{1} zu:
		\begin{align}
			y\textsubscript{1} = \frac{c \textsubscript{y}-a \textsubscript{y}}{c \textsubscript{x}-a \textsubscript{x}} * x
		\end{align}
		Mit m\textsubscript{2}=$\frac{\Delta y}{\Delta x}$ ergibt sich y\textsubscript{2} zu:
		\begin{align}
			y\textsubscript{2} = \frac{c\textsubscript{y}-b\textsubscript{y}}{c\textsubscript{x}-b\textsubscript{x}}*x+b\textsubscript{2}
		\end{align}
		Da die Koordinaten von Eckpunkt B bekannt sind, kann b\textsubscript{2} mit Hilfe von m\textsubscript{2} ermittelt werden.
		\begin{align}
			m\textsubscript{2} = \frac{y\textsubscript{2}-y\textsubscript{1}}{x\textsubscript{2}-x\textsubscript{1}}
		\end{align}
		Mit b\textsubscript{2}=y\textsubscript{2},x\textsubscript{2}=0 und m\textsubscript{2}=$\frac{c\textsubscript{y}-b\textsubscript{y}}{c\textsubscript{x}-b\textsubscript{x}}$ folgt:
		\begin{align}
			b\textsubscript{2} = \frac{(c\textsubscript{y}-c\textsubscript{x})*(c\textsubscript{y}-b\textsubscript{y})}{c\textsubscript{x}-b\textsubscript{x}}
		\end{align}
		Da nun beide Geradengleichungen definiert sind, kann überprüft werden, ob alle Punkte der Bewegung folgende Bedingungen erfüllen:
		\begin{align}
			p\textsubscript{x}<c\textsubscript{x} 
			\land 	p\textsubscript{y}<y\textsubscript{1}(p\textsubscript{x}) 
			\lor
			p\textsubscript{x} \geq c\textsubscript{x} 
			\land 	p\textsubscript{y}<y\textsubscript{2}(p\textsubscript{x})
		\end{align}
		Die Berechnung der räumlichen Begrenzung einer Bewegung im dreidimensionalen Raum kann äquivalent zur Bewegung im Zweidimensionalen durchgeführt werden. Die Begrenzung wird jedoch um folgende Bedingung erweitert:
		\begin{align}
			P\textsubscript{z}<A\textsubscript{z}\land P\textsubscript{z}>0
		\end{align}
		Damit muss nun jeder Punkt einer Bewegung folgende Bedingung erfüllen:
		\begin{align}
			p\textsubscript{x}<c\textsubscript{x} 
			\land 	p\textsubscript{y}<y\textsubscript{1}(p\textsubscript{x})
			\land	P\textsubscript{z}<A\textsubscript{z}
			\land 	P\textsubscript{z}>0 
			\lor
			p\textsubscript{x} \geq c\textsubscript{x} 
			\land 	p\textsubscript{y}<y\textsubscript{2}(p\textsubscript{x})
			\land	P\textsubscript{z}<A\textsubscript{z}
			\land 	P\textsubscript{z}>0
		\end{align}
	\pagebreak