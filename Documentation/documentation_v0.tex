\documentclass[a4paper, 12pt]{article}
\usepackage[a4paper,left=2cm,right=2cm,top=3cm,bottom=3cm]{geometry}
\usepackage[ngerman]{babel}
\usepackage{graphicx}
\usepackage{amsmath}
\usepackage{tikz}
\usepackage{amsmath}
\title{Spidercam}
\date{April 2020}
\author{Andreas Stumpf, Sönke Bartels}
\begin{document}
	\maketitle
	\newpage
	\tableofcontents
	\listoffigures
	\pagebreak
	\section{Einleitung}
	Um automatisierte Kameraführungen in großen und schwer zugänglichen Orten, wie z.B. Stadien, zu ermöglichen werden Kameras an Seilen befestigt.
	In diesem Dokument soll eine solche Kamerasteuerung geplant und durchgeführt werden.
	Dabei wird eine theoretische Herleitung der Ortskoordinaten einer solchen Konstruktion durchgeführt und anschließend an einem Model verifiziert.
	Ziel ist es entweder durch Vorgabe eines Endpunktes eine automatisierte Bewegungssteuerung zum definierten Punkt, oder eine Verfolgung einer Solltrajektorie zu erreichen.
	\pagebreak
	\section{Betrachtung eines Punktes}
		Da eine Darstellung der realen Konstruktion für sämtliche Herleitungen zunächst nicht erforderlich ist, wird im folgenden vereinfacht von der Transformation eines Punktes ausgegangen. Später wird eine Anpassung sämtlicher Herleitungen auf den realen Anwendungsfall erfolgen.
		\subsection{Vereinfachungen im Zweidimensionalen Fall}
		Um das Anfangsproblem zu vereinfachen wird die Herleitung der Längenänderung des Seils eines Motors zunächst im Zweidimensionalen durchgeführt. Das Übersetzen der Funktionen in den dreidimensionalen Raum ist damit trivial.
			\pagebreak
			\subsubsection{Definition der Eckpunkte}
			Um eine Durchführbarkeit innerhalb der geometrischen Begrenzungen zu garantieren, wird ein Koordinatensystem so in den Raum gelegt, sodass sich für zwei der drei Eckpunkte ein x- Wert von Null ergibt. Damit lassen sich Begrenzungsfunktionen zwischen den Eckpunkten aufstellen, die eine einfache Berechnung der Machbarkeit zulassen. Der theoretische Aufbau der Konstruktion sieht dann wie folgt aus:
			\begin{figure}[h!]		
				\begin{tikzpicture}
					\coordinate 	[label=left:0]	(Nullpunkt) at (0,0);
					\coordinate		[label=below:A] 		(A) at (0,0);
					\coordinate		[label=below:B] 		(B) at (14.5,0);
					\coordinate		[label=right:C] 		(C) at (7.25,12);
					\coordinate		[label=right:P] 		(P) at (5,1.5) ;
					\coordinate		[label=right:P`] 		(P`) at (9,5.7);
					\coordinate		[label=below:x]			(X) at (15,0);
					\coordinate		[label=left:y]			(Y) at (0,12.5);
					\coordinate		[label=left:C\textsubscript{y}]	(YC)at (0,12);
					\coordinate		[label=below:C\textsubscript{x}]	(XC)at (7.25,0);
					\coordinate		[label=left:P\textsubscript{y}]	(YP)at (0,1.5);
					\coordinate		[label=below:P\textsubscript{x}]	(XP)at (5,0);
					\coordinate		[label=left:P\textsuperscript{`}\textsubscript{y}]	(YP`)at (0,5.7);
					\coordinate		[label=below:P\textsuperscript{`}\textsubscript{x}]	(XP`)at (9,0);
					\draw [->]	(Nullpunkt) -- (X);
					\draw [->]	(Nullpunkt) -- (Y);
					\draw 	(A) -- (B);
					\draw 	(B) -- (C) node[midway,right] {y\textsubscript{2}};
					\draw 	(C) -- (A) node[midway,left] {y\textsubscript{1}};
					\draw 	(A) -- (P) node[midway,above] {$\vec{a}$};
					\draw 	(A) -- (P`)node[midway,above] {$\vec{a\textsuperscript{'}}$};
					\draw 	(B) -- (P) node[midway,above] {$\vec{b}$};
					\draw 	(B) -- (P`)node[midway,above] {$\vec{b\textsuperscript{'}}$};
					\draw 	(C) -- (P) node[midway,left]  {$\vec{c}$};
					\draw 	(C) -- (P`)node[midway,left]  {$\vec{c\textsuperscript{'}}$};
					\draw 	(P) -- (P`)node[midway,left]  {$\vec{t}$};
					\draw [dashed]  (YP) -- (P);
					\draw [dashed]  (XP) -- (P);
					\draw [dashed]  (YP`) -- (P`);
					\draw [dashed]  (XP`) -- (P`);
					\draw [dashed]  (YC) -- (C);
					\draw [dashed]  (XC) -- (C);
					\fill (A) circle (2pt);
					\fill (B) circle (2pt);
					\fill (C) circle (2pt);
				\end{tikzpicture}
				\caption{Darstellung der Transformation eines Punktes im Zweidimensionalen}
			\end{figure}
			\pagebreak
			\subsubsection{Raumbegrenzung}
			Bevor eine Längenänderung ermittelt werden kann, muss zunächst garantiert werden, dass alle Punkte der Bewegung erreicht werden können. Da zuvor zwei der drei Eckpunkte auf die x- Achse gelegt wurden kann nun einfach ermittelt werden, ob alle Punkte unterhalb der Verbindungsgeraden 
			$\overline {AC}$  
			und 
			$\overline {CB}$
			liegen.
			Dafür werden folgende Geradengleichungen aufgestellt:
			\begin{align}
				y \textsubscript{1} = m \textsubscript{1} * x + b \textsubscript{1}\\
				y \textsubscript{2} = m \textsubscript{2} * x + b \textsubscript{2}
			\end{align}
			Mit b\textsubscript{1}=0 und m\textsubscript{1}=$\frac{\Delta y}{\Delta x}$ ergibt sich y\textsubscript{1} zu:
			\begin{align}
				y\textsubscript{1} = \frac{c \textsubscript{y}-a \textsubscript{y}}{c \textsubscript{x}-a \textsubscript{x}} * x
			\end{align}
			Mit m\textsubscript{2}=$\frac{\Delta y}{\Delta x}$ ergibt sich y\textsubscript{2} zu:
			\begin{align}
				y\textsubscript{2} = \frac{c\textsubscript{y}-b\textsubscript{y}}{c\textsubscript{x}-b\textsubscript{x}}*x+b\textsubscript{2}
			\end{align}
			Da die Koordinaten von Eckpunkt B bekannt sind, kann b\textsubscript{2} mit Hilfe von m\textsubscript{2} ermittelt werden.
			\begin{align}
				m\textsubscript{2} = \frac{y\textsubscript{2}-y\textsubscript{1}}{x\textsubscript{2}-x\textsubscript{1}}
			\end{align}
			Mit b\textsubscript{2}=y\textsubscript{2},x\textsubscript{2}=0 und m\textsubscript{2}=$\frac{c\textsubscript{y}-b\textsubscript{y}}{c\textsubscript{x}-b\textsubscript{x}}$ folgt:
			\begin{align}
				b\textsubscript{2} = \frac{(c\textsubscript{y}-c\textsubscript{x})*(c\textsubscript{y}-b\textsubscript{y})}{c\textsubscript{x}-b\textsubscript{x}}
			\end{align}
			Da nun beide Geradengleichungen definiert sind, kann überprüft werden, ob alle Punkte der Bewegung folgende Bedingungen erfüllen:
			\begin{align}
						p\textsubscript{x}<c\textsubscript{x} 
				\land 	p\textsubscript{y}<y\textsubscript{1}(p\textsubscript{x}) 
				\lor
						p\textsubscript{x} \geq c\textsubscript{x} 
				\land 	p\textsubscript{y}<y\textsubscript{2}(p\textsubscript{x})
			\end{align}
			\subsubsection{Berechnung der Längenänderungen}
			Ist garantiert, dass alle Punkte der Bewegung realisierbar sind kann die Berechnung der Seillängenänderung erfolgen. 
			Die Längenänderung der Steuerseile kann geometrisch bestimmt werden.
			Dabei wird davon ausgegangen, dass die aktuelle Kameraposition P, sowie die Montagepositionen der Aktuatoren A, B und C bekannt sind.\\
			Die Länge des Seils a zu Motor A kann bestimmt werden als Subtraktion des Vektors zu Motor A und des Vektors zu P.\\
			Die Länge des Seils a' zu Motor A kann bestimmt werden als Subtraktion des Vektors zu Motor A und des Vektors zu P'.\\
			Damit ergibt sich die Längenänderung des Seils zu Motor A als Differenz der Längen a und a'.
			Analog dazu ergeben sich die Längenänderungen der Seile zu Motor B und C.
			Soll ein Punkt mit drei Aktuatoren in der Ebene bewegt werden,
			so ergibt sich die Längenänderung $\Delta n$ in Abhängigkeit der Montageposition des Aktuators und der angestrebten Kameraposition zu:
			\begin{align}
				\Delta n =
				|\vec{n^{'}}|-|\vec{n}|=
				\sqrt{
				(P^{'}\textsubscript{x}-N\textsubscript{x})^2+
				(P^{'}\textsubscript{y}-N\textsubscript{y})^2
				}
			\end{align}
		\subsection{Der Dreidimensionale Fall}	
			\subsubsection{Definition der Eckpunkte}
			Die Positionen der Motoren bleiben nach der Betrachtung im Zweidimensionalen identisch, jedoch wird jeder Motorposition eine Z- Koordinate hinzugefügt, sodass sich die Eckpunkte nun im Format N=(N\textsubscript{x} N\textsubscript{y} N\textsubscript{z}) befinden. Die x- und y- Komponenten bleiben bestehen, um sämtliche Vereinfachungen aus dem Zweidimensionalen übernehmen zu können.
			\begin{figure}[h!]	
				\begin{tikzpicture}[scale=1]		   
				\coordinate 	[label=left:A] 										(A) 	at 	(0,10,0);
				\coordinate 	[label=right:B] 									(B) 	at 	(14.5,10,0);
				\coordinate 	[label=above:C] 									(C) 	at 	(7.25,10,-12);
				\coordinate 	[label=left:P] 										(P) 	at 	(5,3.5,-1.5);
				\coordinate 	[label=right:P'] 									(P') 	at 	(9,3.5,-5.7);
				
				\coordinate		[label=left:P\textsubscript{y}]						(YP)	at 	(0,0,-1.5);
				\coordinate		[label=below:P\textsubscript{x}]					(XP)	at 	(5,0,0);
				\coordinate		[label=left:P\textsuperscript{`}\textsubscript{y}]	(YP')	at 	(0,0,-5.7);
				\coordinate		[label=below:P\textsuperscript{`}\textsubscript{x}]	(XP')	at 	(9,0,0);
				\coordinate		[label=below:B\textsubscript{x}]					(XB)	at 	(14.5,0,0);
				\coordinate		[label=below:C\textsubscript{x}]					(XC)	at 	(7.25,0,0);
				\coordinate		[label=left:C\textsubscript{y}]					    (YC)	at 	(0,0,-12);
				
				%(x		,z		,y)
				\draw[thick,->] (0,0,0) 		-- 	(15,0,0) 		node[anchor=north west]	{$x$};
				\draw[thick,->] (0,0,0) 		-- 	(0,0,-15) 		node[anchor=south east]	{$y$};
				\draw[thick,->] (0,0,0) 		-- 	(0,15,0) 		node[anchor=south]		{$z$};	
				\draw[thick] 	(A) 			-- 	(B) 			;		
				\draw[thick] 	(B) 			-- 	(C) 			node[midway,right]		{$y2$};	
				\draw[thick] 	(C) 			-- 	(A) 			node[midway,above]		{$y1$};	
				\draw[thick] 	(A) 			-- 	(P) 			node[midway,above]		{$a$};
				\draw[thick] 	(A) 			-- 	(P') 			node[midway,above]		{$a'$};
				\draw[thick] 	(B) 			-- 	(P) 			node[midway,above]		{$b$};
				\draw[thick] 	(B) 			-- 	(P') 			node[midway,right]		{$b'$};
				\draw[thick] 	(C) 			-- 	(P) 			node[midway,right]		{$c$};
				\draw[thick] 	(C) 			-- 	(P') 			node[midway,left]		{$c'$};
				\draw[thick] 	(P) 			-- 	(P') 			node[midway,above]		{$t$};
				\draw[dashed,-]  (XB) 			-- 	(B) 		;
				\draw[dashed,-]  (XC) 			-- 	(7.25,0,-12) 		;
				\draw[dashed,-]  (YC) 			-- 	(7.25,0,-12) 		;
				\draw[dashed,-]  (7.25,0,-12) 	-- 	(C) 		;
				\draw[dashed,-]  (XP) 			-- 	(5,0,-1.5) 		;
				\draw[dashed,-]  (YP) 			-- 	(5,0,-1.5) 		;
				\draw[dashed,-]  (5,0,-1.5) 	-- 	(P);
				\draw[dashed,-]  (XP') 			-- 	(9,0,-5.7) 		;
				\draw[dashed,-]  (YP') 			-- 	(9,0,-5.7) 		;
				\draw[dashed,-]  (9,0,-5.7) 	-- 	(P');
				\end{tikzpicture}
				\caption{Darstellung der Transformation eines Punktes im Dreidimensionalen} 
			\end{figure}
		\pagebreak
			\subsubsection{Raumbegrenzung}
			Die Berechnung der räumlichen Begrenzung einer Bewegung im dreidimensionalen Raum kann äquivalent zur Bewegung im Zweidimensionalen durchgeführt werden. Die Begrenzung wird jedoch um folgende Bedingung erweitert:
			\begin{align}
			P\textsubscript{z}<A\textsubscript{z}\land P\textsubscript{z}>0
			\end{align}
			Damit muss nun jeder Punkt einer Bewegung folgende Bedingungen erfüllen:
			\begin{align}
			p\textsubscript{x}<c\textsubscript{x} 
			\land 	p\textsubscript{y}<y\textsubscript{1}(p\textsubscript{x})
			\land	P\textsubscript{z}<A\textsubscript{z}
			\land 	P\textsubscript{z}>0 
			\lor
			p\textsubscript{x} \geq c\textsubscript{x} 
			\land 	p\textsubscript{y}<y\textsubscript{2}(p\textsubscript{x})
			\land	P\textsubscript{z}<A\textsubscript{z}
			\land 	P\textsubscript{z}>0
			\end{align}
			\subsubsection{Berechnung der Längenänderungen}
			Die Berechnung der Längenänderung ergibt sich analog zum Zweidimensionalen, wird jedoch um die Z- Komponente erweitert.
			\begin{align}
			\Delta n =
			|\vec{n^{'}}|-|\vec{n}|=
			\sqrt{
				(P^{`}\textsubscript{x}-N\textsubscript{x})^2+
				(P^{`}\textsubscript{y}-N\textsubscript{y})^2+
				(P^{`}\textsubscript{z}-N\textsubscript{z})^2
			}
			\end{align}	
\end{document}