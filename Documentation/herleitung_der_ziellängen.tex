\section{Herleitung der Ziellängen bzw. der Längenänderungen}
	Im Folgenden werden sämtlich zuvor getroffenen Annahmen genutzt um die Ziellängen bzw. die Längenänderung aller Seile allgemein zu errechnen.
		\subsection{Ziellängen}
		Um eine xxx Regelung umzusetzen werden hier nun die Ziellängen aller Seile allgemein errechnet, die erreicht werden müssen um die Kamera an den gewünschten Punkt zu befördern.
		\subsection{Längenänderung}
			Die Berechnung der Ziellängen wird erweitert, um genaue Längenänderungen aller Motoren zu erhalten, welche notwendig sind um eine xxx Regelung umzusetzen.
			Ist garantiert, dass alle Punkte der Bewegung realisierbar sind kann die Berechnung der Seillängenänderung erfolgen.Die Längenänderung der Steuerseile kann geometrisch bestimmt werden.
			Dabei wird davon ausgegangen, dass die aktuelle Kameraposition P, sowie die Montagepositionen der Aktuatoren A, B und C bekannt sind.Die Länge des Seils a zu Motor A kann bestimmt werden als Subtraktion des Vektors zu Motor A und des Vektors zu P.Die Länge des Seils a' zu Motor A kann bestimmt werden als Subtraktion des Vektors zu Motor A und des Vektors zu P'.Damit ergibt sich die Längenänderung des Seils zu Motor A als Differenz der Längen a und a'.Analog dazu ergeben sich die Längenänderungen der Seile zu Motor B und C.
			Soll ein Punkt mit drei Aktuatoren in der Ebene bewegt werden,so ergibt sich die Längenänderung $\Delta n$ in Abhängigkeit der Montageposition des Aktuators und der angestrebten Kameraposition zu:
			\begin{align}
				\Delta n =
				|\vec{n^{'}}|-|\vec{n}|=
				\sqrt	{
						(P^{'}\textsubscript{x}-N\textsubscript{x})^2+
						(P^{'}\textsubscript{y}-N\textsubscript{y})^2
						}
			\end{align}
			Die Berechnung der Längenänderung im Dreisimensionalen ergibt sich analog zum Zweidimensionalen, wird jedoch um die Z- Komponente erweitert.
			\begin{align}
				\Delta n =
				|\vec{n^{'}}|-|\vec{n}|=
				\sqrt	{
						(P^{`}\textsubscript{x}-N\textsubscript{x})^2+
						(P^{`}\textsubscript{y}-N\textsubscript{y})^2+
						(P^{`}\textsubscript{z}-N\textsubscript{z})^2
						}
			\end{align}	
			Die Berechnung der Längenänderung mit Plattform muss um die auf Seitenlänge normierten Richtungsvektoren der Plattform erweitert werden.
			\begin{align}
				\Delta n =
				|\vec{n^{'}}|-|\vec{n}|=
				\sqrt	{
						(P^{`}\textsubscript{x}-N\textsubscript{x})^2+
						(P^{`}\textsubscript{y}-N\textsubscript{y})^2+
						(P^{`}\textsubscript{z}-N\textsubscript{z})^2
						}
			\end{align}
	\pagebreak